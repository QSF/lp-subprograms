%!TEX root = ../artigo.tex
\section{Chamar Subprogramas Indiretamente} % (fold)
\label{sec:chamar_subprogramas_indiretamente}
Há momentos, durante a programação de um software, em que se torna necessário chamar subprogramas de forma indireta. Isso ocorre quando o subprograma a ser chamado é somente conhecido em tempo de execução, como eventos disparados por bibliotecas de interface gráfica e funções de \textit{callback}.

%exemplo em c/c++
Em C e C++, podemos utilizar ponteiro para função para chamar um subprograma conhecido em tempo de execução \cite{pointer,sebesta}. Para utilizar essa técnica, primeiro temos que declarar uma função, como por exemplo:
\begin{verbatim}
int sum(int a, int b)
{
    return a + b;
}
\end{verbatim}

A seguir, temos que declarar um ponteiro com a mesma assinatura da função escolhida. Como no nosso exemplo (função \textit{sum}) a função possui dois parâmetros e um retorno do tipo \emph{int}, temos um ponteiro como mostrado:
\begin{verbatim}
int (*sum_pointer)(int, int);
\end{verbatim}

Em seguida, é necessário atribuir a função em questão para o ponteiro declarado. Em nosso exemplo:
\begin{verbatim}
sum_pointer = &sum;
\end{verbatim}

Por fim, basta invocar a função, como da seguinte maneira:
\begin{verbatim}
(*sum_pointer)(1,2);	
\end{verbatim}

%exemplo em C#
Em C\#, podemos referenciar métodos em forma de objetos, através do uso de \emph{delegate} \cite{delegate,sebesta}, o que torna muito poderoso e flexível. Para fazer uso do mesmo, precisamos declarar um \emph{delegate} para um determinado protocolo de função, como:
\begin{verbatim}
public delegate int SumDelegate(int a, int b);
\end{verbatim}

Podemos instanciar um SumDelegate passando por parâmetro para seu construtor o nome de uma função cuja declaração tenha o mesmo protocolo. Suponha a existência da função chamada \textit{sum} que respeite essa assinatura, podemos então instanciar SumDelegate como segue:
\begin{verbatim}
SumDelegate sumDelegate= new SumDelegate(sum);
\end{verbatim}

Para executar o \textit{delegate} em questão, fazemos da seguinte forma:
\begin{verbatim}
sumDelegate(2,3);
\end{verbatim}
% section chamar_subprogramas_indiretamente (end)