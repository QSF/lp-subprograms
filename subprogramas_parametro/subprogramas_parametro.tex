%!TEX root = ../artigo.tex

\section{Subprogramas Como Parâmetro} % (fold)
\label{sec:subprogramas_como_parametro}
Em muitas ocasiões temos a necessidade de passar um subprograma através de um parâmetro. A ideia é interessante e simples, mas gera duas complicações em termos de implementação.

Primeiro, temos a complicação que consiste na maneira de realizar a checagem de tipo (\textit{type checking}, ~\ref{sub:type_check}) do subprograma passado por parâmetro. Em C e C++, onde a passagem de subprogramas é feita através de ponteiro para função, essa checagem é feita pelo tipo do ponteiro.

A outra complicação ocorre em linguagens de programação que permitem subprogramas aninhados. O problema refere-se a qual ambiente de referência o subprograma passado por parâmetro terá. Nessa situação, há três tipos possíveis:
\begin{description}
	\item[Shallow Binding:] O ambiente é o local onde o subprograma é chamado.
	\item[Deep Binding:] O ambiente refere-se onde o subprograma foi definido.
	\item[Ad Hoc Binding:] O ambiente condiz com o local que o subprograma foi passado por parâmetro.
\end{description}

Como exemplo, considere a listagem ~\ref{scope_binding}, cuja sintax é de JavaScript. O subprograma \textit{sub2()} apenas imprime o valor da variável \textit{x}, porém, seu valor depende do ambiente de referência utilizado. 

\begin{lstlisting}[caption=Código retirado de \cite{sebesta}]
function sub1() {
	var x;
	function sub2() {
		alert(x);
	};
	function sub3() {
		var x;
		x = 3;
		sub4(sub2);
	};
	function sub4(subx) {
		var x;
		x = 4;
		subx();
	};
	x = 1;
	sub3();
};
\end{lstlisting}
\label{scope_binding}

Caso a listagem em questão utilize o ambiente de referência \textit{Shallow Binding}, o valor impresso seria 4. Caso o ambiente \textit{Deep Binding} fosse escolhido, o valor impresso seria 1. Já para \textit{Ad Hoc Binding}, o valor seria 3.

Segundo Robert W. Sebesta, \textit{Ad Hoc Binding} nunca foi implementado.
% section subprogramas_como_parametro (end)