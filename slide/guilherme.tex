%!TEX root = slide.tex

\section{Fundamentos} % (fold)
\label{sec:introducao}

\begin{frame}{Introdução}
	\begin{itemize}
	  \item Conceito muito importante nas linguagens de programação.
	  \item Máquina análitica de Babbage
	  \item Reuso, economia de tempo e abstração.
	  \item Métodos nas linguagens orientadas a objeto também são subprogramas
	\end{itemize}
\end{frame}

\begin{frame}{Características comuns}
	\begin{itemize}
		\item Cada subprograma tem um único ponto de entrada.
		\item Há apenas um subprograma em execução em um dado momento.
		\item O controle sempre retorna para a estrutura que chamou quando a execução do subprograma termina.
	\end{itemize}
\end{frame}

\begin{frame}[fragile]{Cabeçalho de subprograma}

	Fornece um nome para o subprograma e especifica uma lista de parâmetros.

	\begin{itemize}
		\item Ruby e Python
		\begin{verbatim}
			def funcao (parametros)
		\end{verbatim}

		\item C-based
		\begin{verbatim}
			void funcao (parametros)
		\end{verbatim}

	\end{itemize}
\end{frame}

\begin{frame}{Corpo dos subprogramas}

	Os corpos dos subprogramas definem as computações.

	\begin{itemize}
		\item \emph{C-based.} Delimitadas por chaves \{ \}
		\item \emph{Python.} Identação
		\item \emph{Ruby.} Palavra-chave \textbf{end}
	\end{itemize}

\end{frame}

\begin{frame}[fragile]{Peculiaridade de Ruby}
    
	\begin{lstlisting}[language=ruby]
class Exemplo
  def invocar_subprograma
    puts self.method(:invocar_subprograma).owner #Exemplo
    subprograma
  end
end

def subprograma
  puts self.method(:subprograma).owner # Object
end

exemplo = Exemplo.new
exemplo.invocar_subprograma
	\end{lstlisting}

\end{frame}

\begin{frame}{Parâmetros}
	
	\begin{itemize}
		\item Parâmetros Reais (Argumentos)
		\item Parâmetros Formais (Parâmetros)
	\end{itemize}

\end{frame}

\begin{frame}[fragile]{Exemplos de parâmetros por palavras-chave}
	
	\begin{itemize}

		\item Parâmetros Formais (Parâmetros)
		\begin{verbatim}
			funcao(20, 10)
		\end{verbatim}

		\item Parâmetros Reais (Argumentos)
		\begin{verbatim}
			void funcao(int param1, int param2)
		\end{verbatim}
	\end{itemize}

\end{frame}

\begin{frame}[fragile]{Exemplos de parâmetros por palavras-chave}

 Ada, Fortran 95+ e Python     
	\begin{lstlisting}[language=python]
def subprograma(param1, param2, param3):
  print param1 # 30
  print param2 # 20
  print param3 # 10

subprograma(param3 = 10, param2 = 20, param1 = 30)

	\end{lstlisting}
Desvantagem: Cliente precisa saber o nome dos parâmetros
\end{frame}

\begin{frame}[fragile]{Exemplos de parâmetros com valores padrão}

Python, Ruby, C++, Fortran 95+, Ada e PHP

	\begin{lstlisting}[language=python]
def subprograma(param1, param2 = 20, param3 = 30):
  print param1 # 10
  print param2 # 20
  print param3 # 30

subprograma(10)

	\end{lstlisting}

\end{frame}

\begin{frame}[fragile]{Passagem de hash e listas como parâmetros}


	\begin{lstlisting}[language=ruby]
class Conta
  def transfere(argumentos, *valores)
    destino = argumentos[:para]
    data = argumentos[:em]

  end
end

conta = Conta.new
conta.transfere({:para => :escola, :em => Time.now}, [20.0, 30.0])


	\end{lstlisting}

\end{frame}
