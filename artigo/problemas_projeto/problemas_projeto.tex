%!TEX root = ../artigo.tex
\section{Projeto de Subprogramas} 
\label{sec:projeto_de_subprogramas}
Subprogramas são estruturas complexas em linguagens de programação 
e é considerada um dos grandes problemas em seu projeto.
Um problema comum é a escolha de ou mais méotdos de
passagem de parâmetro que será utilizado \cite{sebesta}. 

A natureza do ambiente local de um subprograma dita as regras da natureza do 
próprio subprograma. A questão mais importante é se variáveis locais serão alocadas 
estaticamente ou dinamicamente \cite{sebesta}.

Finalmente, o projetista precisa analisar se os
subprogramas podem ser sobrecarregados ou genéricos.
Um subprograma sobrecarregado é aquele que tem o mesmo nome de outro 
subprograma no mesmo ambiente referenciado e um programa genérico
é aquele cuja computação pode ser feita dos dados de tipos 
diferentes em chamadas diferentes \cite{sebesta}.