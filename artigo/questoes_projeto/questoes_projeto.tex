\section{Questões de projetos referente a funções} 
\label{sec:questoes_de_projetos_referentes_a_funcoes}
Algumas questões relevantes ao projeto em relação a funções precisam ser levados em conta. As próximas seções discutem sobre questões de efeitos colaterais, tipos de retorno e quantidade de valores de retorno.

\subsection{Efeitos colaterais}
\label{sub:efeitos_colaterais}
Uma questão relevante ao projeto em relação a funções é se efeitos colaterais serão permitidos. Linguagens puramente funcionais como Haskell, não possuem variáveis e portanto suas funções não possuem efeitos colaterais. Já linguagens que permitem funções com parâmetros passados por valor ou por referência, possuem efeitos colaterias, como por exemplo \textit{aliasing} (\textit{alias}, apelido). Considere o exemplo a seguir.

\begin{verbatim}
int x = 3; 
... // se int* y = &x;
*y = 9;
\end{verbatim}

O compilador nada pode fazer para otimizar o código acima, uma vez que não é possível saber se y = \&x. Então, se o otimizador tentar aplicar, por exemplo, o constant propagation sabendo que x é 5, e y for um \textit{alias}, ponteiro para x, então a propagação será feita de forma errada. 

\subsection{Tipos de Valores Retornados}
\label{sub:tipos_de_valores_retornados}
C permite qualquer tipo ser retornado por suas funções exceto vetores e funções. Ambos nesse caso podem ser manipulados por ponteiros. C++ é semelhante ao C mas também permite tipos definidos pelo usuário ou classes serem retornados. Fortran 77 e Pascal permitem apenas tipos não estruturados. Ada, Python, Ruby e Lua, retornam valores de qualquer tipo, exceto no caso do Ada, onde funções não são tipos, portanto não podem ser retornadas (mas ponteiros para funções sim). Em Java e C\#, qualquer tipo ou classe podem ser retornados por seus métodos, porém, como método não é um tipo, não pode ser retornado.

\subsection{Quantidade de valores retornados}
\label{sub:quantidade_de_valores_retornados}
A maioria das linguagens atuais permitem apenas 1 valor de retorno. A linguagem Lua, permite o retorno de multiplos valores. Considere um exemplo de função na linguagem Lua que retorna 3 valores \cite{sebesta}. A ordem das variáveis que chamam a função, serão vinculadas na ordem dos valores retornados pela função. \\

\noindent A chamada da função

\begin{verbatim}
a, b, c = fun() 
\end{verbatim}
Onde o retorno da função fun() é 
\begin{verbatim}
return 3, sum, index
\end{verbatim}

