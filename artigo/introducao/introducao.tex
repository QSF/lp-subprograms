%!TEX root = ../artigo.tex

\section{Introdução} 
\label{sec:introducao}
Abstração de processos, na forma de subprogramas, foram 
um conceito central em todas as linguagens de programação.
O primeiro computador programável, a máquina análitica de Babbage \cite{babbage1910babbage},
reusava coleções de cartões de instruções em diferentes partes do programa.
Em uma linguagem de programação moderna, tal reuso se dá através de subprogramas
e economiza espaço em memória e tempo de codificação e, além disso,
abstrai a implementação de um procedimento através de uma declaração de chamada.
Isso melhora a legibilidade do programa, o que enfatiza sua estrutura lógica
enquanto esconde os detalhes de baixo nível \cite{sebesta}.

Os métodos das linguagens orientados a objeto estão intimamente ligados 
aos subprogramas, com a principal diferença das chamadas de método
estarem associadas com classes e objetos \cite{sebesta}.

