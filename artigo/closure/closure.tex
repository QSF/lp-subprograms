\section{Closure}
\label{sec:closure}
Closure é uma variável local em uma função que é mantida viva (não é desalocada) após o retorno dessa função. Apenas linguagens que permitem subprogramas aninhados é possivel closure. Linguagens como C\# e JavaScript possuem closure. Considere a listagem \ref{lst:closure}, na linguagem JavaScript, que utiliza closure.

\begin{lstlisting}[caption={Closure em JavaScript}, label={lst:closure}]
function foo(x) {
  var tmp = 3;
  return function (y) {
    alert(x + y + (++tmp));
  }
}

var bar = foo(2);
bar(10); 
\end{lstlisting}

\noindent
No programa acima na linha 8, bar é um closure. A saída do programa exibe 16 como saída quando executado pela primeira vez, e vai incrementando em 1 a cada execução.